\documentclass{article}

\usepackage{hyperref}

\hypersetup{
   pdftitle    ={Openbook Filesystem},%
   pdfauthor   ={Joshua Bialkowski, Bruno Alvisio, Gabe Ayers},%
   pdfsubject  ={},%
   pdfcreator  ={},%
   pdfproducer ={},%
   pdfkeywords ={6.824},
   colorlinks=true,
   linkcolor=red,
   anchorcolor=black,
   citecolor=green,
   urlcolor=magenta,
   plainpages=false,
   hypertexnames=true
}

\title
{%
   Openbook Filesystem \\
   \small 6.824 Project Proposal
}

\author
{%
   \small Josh Bialkowski \\
   \small Bruno Alvisio \\
   \small Gabe Ayers (potentially)
}



\begin{document}
\maketitle

We propose to make a free-software file synchronizer based on some of the ideas 
behind Ficus and Bayou. The main idea is to utilize File Systems in User Space
(FUSE) to run the synchronization system in the background and as a user 
process. While a truly distributed architecture would be the end-goal, for the
scope of this project we plan to implement a system that relies on a dedicated
primary server. We are attempting to design the system in such a way as to make
the eventual transition to a truly distributed architecture managable in the
future. 

The software will be written in C++. We intend to focus on building the project
with only free-software (GPL compatable) components and to build the system 
from the ground up with a focus on security and usability. Our implementation 
will focus primarily on the availability of files with the goal of eventual
consistency. The project has a public face on github at 
\url{https://github.com/cheshirekow/openbookfs}.

We will also be developing a user interface that will manage multiple different
mount points as well as creating an area for the user to override/manage conflict
resolution. Most conflict resolution will be attempted automatically but will be
passed on to the user if the conflict can not be resolved automatically.

\end{document}
