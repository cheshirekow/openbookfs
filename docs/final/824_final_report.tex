\documentclass[10pt,twocolumn]{article}
\title{ \vspace{-3cm} Openbook File System \vspace{-0.5cm} }
\author{ Bruno Alvisio, Gabe Ayers, Josh Bialkowski}


\usepackage{fullpage}
\usepackage{amsfonts}
\usepackage{amssymb}
\usepackage{amsmath}
\usepackage{hyperref}
\usepackage[compact]{titlesec}
\usepackage{flushend}

\hypersetup{
    bookmarks=true,         % show bookmarks bar?
    unicode=false,          % non-Latin characters in Acrobat�s bookmarks
    pdftitle={Openbook FS}, % title
    pdfauthor={Author},     % author
    pdfsubject={Subject},   % subject of the document
    pdfcreator={Creator},   % creator of the document
    pdfproducer={Producer}, % producer of the document
    pdfkeywords={keyword1} {key2} {key3}, % list of keywords
    pdfnewwindow=true,      % links in new window
    colorlinks=true,        % false: boxed links; true: colored links
    linkcolor=red,          % color of internal links (change box color with linkbordercolor)
    citecolor=green,        % color of links to bibliography
    filecolor=magenta,      % color of file links
    urlcolor=cyan           % color of external links
}


\titlespacing{\section}{0pt}{*0}{5pt}
\titlespacing{\subsection}{0pt}{5pt}{*0}
\titlespacing{\subsubsection}{0pt}{*0}{*0}

\setcounter{secnumdepth}{1}

\begin{document}
\maketitle
\section{Introduction}

There are many pieces of software available providing version file 
synchronization: i.e, Google Drive, Drop Box, Spiderbox, Syncplicity, Cubby,
Huddle, Egnyte, ShareFile, etc. However, there are inherent privacy and 
security concerns with using a software-as-a-service proprietary solution. 
Furthermore, while a centralized service provides a useful backup location, it
provides an artificial requirement for synchronization which prevents 
peer-to-peer sync for machines connected to each other on a network but not
to the internet.

We seek to address these issues by building a free-software (GPL) peer-to-peer
file synchronizer, which we haved dubbed Openbook File System (OBFS). OBFS
follows the synchronization model of Ficus, but works entirely in user space. 
This document briefly describes the architecture of the underlying system, 
the user interfaces, what has been accomplished for the 6.824 
proof-of-concept release, and what are goals are for the future of the code.  

\section{Architecture}

OBFS seeks to achieve \emph{eventual consistency} and immediate availability 
of local files. We achieve eventual consistency by identifying causal orderings
and automatically transfering files that are strictly future versions. 

\subsection{Causal Ordering with Version Vectors}
OBFS determines causal orderings of file changes by the use of version
vectors. Every file is assigned a version vector which is a map from peer id to
version number. When the user changes a file, the version element for appropriate
peer id is incremented by one. Version vectors induce a partial order on events
and allow us to approximate causal relationships. When two peers synchronize
they send to each other the version vector for their copy of a particular file.
If the version vectors are ordered, then the peer with the lesser version
downloads the version from the other. If they are not ordered, the file is noted
to be in conflict. The user must intervene to resolve conflicts.

\subsection{File System Interface}
OBFS monitors the file system for file changes by installing a FUSE 
\emph{overlay} file system. OBFS can expose any of the usual Linux file systems
(for instance, ext3, ext4, etc.). The user interacts only with the mount point,
not with the underlying file system. Files and directories are stored as usual
in the underlying file system, with the exception that every directory contains
an additional SQLite database file which stores meta data for that directory
(including version vectors for the directory and for all files in that 
directory). Most file system events that are passed to the FUSE layer are
serviced directly by the underlying file system except that any events which
signify a change in the file increment the version number for that machine's
key in the version vector and potentially send a message to other peers. 

OBFS peers always synchronize the directory structure and file name information, but actual file contents are only synchronized between peers who have
``checked out'' the file. In addition, the only peers who contribute elements 
to the version vector are peers who have checked out the file. This allows
users to effectively manage the distribution of large files across multiple 
machines with varying storage capacity. 

\subsection{Peer-to-peer communication}
OBFS uses long lived TCP connections between peers. When two peers are connected,
any file modification event intercepted by the FUSE layer generates a version
advisory message which is passed to any connected peers with entries in the 
version vector. In addition to these messages, version messages can also be
sent during a ``full sync'' requested by the user. A full sync navigates the
entire file tree and exchanges version information for every checked out file. 
This is the manner in which a peer discovers that another peer has checked out
a file, and elminates the need to queue messages to disconnected peers 
(which may remain disconnected indefinitely). It is our plan to handle
intermittent disconnections by queueing messages for a short time after a 
peer disconnects, so that if they reconnect a full sync is not required to
achieve consistency. 


\subsection{Conflict Resolution}
Given two peers $A$ and $B$, for a particular file $f$, let $v_x(f)$ denote the
version vector for file $f$ on peer $x$. We denote the file on $A$ relative to
$B$ as either `synced', `stale', `fresh', or in `conflict'. See Table \ref{tbl:FileRelationship}.

\begin{table}
    \centering
    \begin{tabular}{r|l|l}
        relationship        &  $f_A$ is\ldots & $f_B$ is\ldots \\ \hline
        $v_A(f) = v_B(f)$   &  synced         & synced         \\
        $v_A(f) < v_B(f)$   &  stale          & fresh          \\
        $v_A(f) > v_B(f)$   &  fresh          & stale          \\
        incomparable        &  conflict       & conflict       
    \end{tabular}
    \caption{File relationship between peers}
	\label{tbl:FileRelationship}
\end{table}

While mechanisms for automatic conflict resolution (as in Ficus and Bayou) are
an eventual goal of OBFS, at this time there is no mechanism implemented. Rather
when a file is in conflict, the user is alerted and provided with the option of
choosing which of the two versions to keep and which to discard. The user is
also alerted if he attempts to open a stale file for writing (in order to 
prevent a file going into conflict). 

File transfers are performed in a separate thread and occur on a lower 
priority communication channel (version change events may preempt file
transfers). File transfers are queued whenever a file is found to be either
stale or in conflict. If a file transfer completes for a stale file, the stale
file is automatically replaced. If a file transfer completes for a conflicted 
file, the user is given the opportunity to open both versions in order to compare 
them. 

\subsection{Security}

OBFS uses RSA public keys to uniquely identify peers in the network, using an
RSA encryption challenge to verify authenticity. All traffic between peers is
AES encrypted using a one-time content encryption key determined through Diffie
Helman key exchange. The FUSE module is capable of mounting over encrypted file
systems to provide a secure storage backend (i.e., the mount chain ext4 
$\rightarrow$ ecryptfs $\rightarrow$ openbookfs).



\section{User Interfaces}
We've built OBFS to be as extensible as possible. The backend process listens
not only for peer connections, but also for user interface connections. User
interface connections are priviledged local connections over which messages
for configuring and manipulating the backend are exposed. 

We have built two user interfaces that are capable of connecting to the 
backend: one command line interface and one graphical interface. Both 
interfaces allow the user to change configuration parameters, list, add, and
remove mount points, list and authorize known peers, connect to peers, 
and checkout and release files. 

\section{Interesting Problems}

We ran into many interesting problems in building this proof of concept. For the
sake of brevity we will only highlight one of these here: We originally 
implemented message handling so that when a message arrives it was dispatched
to a pool of workers. We assumed that messages would be more or less independant
and that this would be a simple way of allowing the system to scale with peer
connections and to take advantage of any available processing power on the 
users machine. However, as should have been obvious, file 
system events are not independant, and many seemingly atomic actions by the
user result in a sequence of several calls to the underlying file system. 
In particular, when opening a text file in \texttt{gedit}, changing some text, 
and then saving the file, the resulting events that generate messages are in 
fact:
(1) create a temporary file of 0 size, (2) truncate that file to the correct
size, (3) write the data, (4) move the file to overwrite the original. When 
these events generated messages they were being processed out of order by the
connected peer which would crash the backend. We fixed this problem by
assigning a single worker to each connection and processing messages 
sequentially in the order they are received (which, with TCP, means the order
they were sent).  



\section{Conclusion}

For this 6.824 project we have completed a working proof of concept with 
directory and file synchronization, conflict identification, and user
conflict resolution over the UI. The high level features completed and to do
are summarized in Table \ref{tbl:FeatureList}.


\begin{table}[ht]
    \centering
    \begin{tabular}{l|l}
        \textbf{completed}      &   \textbf{to do}  \\ \hline
        RPC between peers       &   Panel UI        \\
        Encryption              &   Hazard Alerts   \\
        Fuse overlay            &   Maintain some history   \\
        Version vectors         &   Auto conflict resolution   \\
        Full synchronization    &   Peer discovery          \\
        Event synchronization   &\\
        Command line UI         &\\
        Graphical UI            
    \end{tabular}
    \caption{Feature list}
    \label{tbl:FeatureList}
\end{table}

The code can be found on GitHub. Our three GitHub accounts and links to 
our repositories are shown in Table \ref{tbl:github}. 

\begin{table}[ht]
    \centering
    \begin{tabular}{l|l}
        \textbf{name}     &  \textbf{respository}    \\ \hline
        Bruno Alvisio     &  \url{balvisio/openbookfs}    \\
        Gabe  Ayers       &  \url{gabeayers/openbookfs}   \\
        Josh  Bialkowski  &  \url{cheshirekow/openbookfs} \\
    \end{tabular}
    \caption{GitHub repositories for team members. Note: the coordination is 
             performed on Josh's account}
    \label{tbl:github}
\end{table}


The code is written in C++. We use the CMake build system, and OBFS depends on
the packages in Table \ref{tbl:dependencies}.


\begin{table}[ht]
    \centering
    \begin{tabular}{l|l}
      \textbf{library} & \textbf{usage} \\ \hline
      crypto++          & cryptography        \\
      protocol buffers  & message marshalling \\
      yaml-cpp          & configuration files \\
      sqlite3           & meta data storage   \\
      soci              & sqlite3 interface   \\
      RE2               & regular expression  \\
      posix threads     & threads, mutex      \\
      fuse              & file system         \\
      tclap             & command line        \\
      boost             & file system interface\\ 
     Qt		    & graphic user interface
    \end{tabular}
    \caption{OBFS dependencies}
    \label{tbl:dependencies}
\end{table}


\end{document}
